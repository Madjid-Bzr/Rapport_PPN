
\section{Introduction}

Dans le monde de la programmation informatique, l’optimisation est le processus qui consiste à réduire le temps d'exécution d’une fonction, l’espace occupé par les données et le programme, ou la consommation d'énergie.\\

En règle générale, l'optimisation doit s’effectuer une fois que le programme est fonctionnel et qu’il réponde aux spécificités attendues.\\

Avant de commencer l'optimisation, pour cela, il existe plusieurs approches d’optimisation l’une plus complexe que l’autre, on peut citer quelques unes :

\begin{itemize}
    \item au niveau algorithmique, en choisissant un algorithme de complexité inférieure (au sens mathématique) et des structures de données adaptées,
    
    \item au niveau du langage de développement, en ordonnant au mieux les instructions et en utilisant les bibliothèques disponibles,
    
    \item en utilisant localement un langage de bas niveau, qui peut être le langage C ou, pour les besoins les plus critiques, le langage assembleur.

\end{itemize}

	Comme on peut le voir tout cela demande enormement de temps et de travail. C’est là qu'intervient maqao (Modular Assembly Quality Analyzer and Optimizer) qui est de façon générale un ouitl d'analyse et d'optimisation des performances.\\
	
	Dans le premier chapitre ci-dessous, nous allons définir plus en détails ce qu'est MAQAO.
	
