
\section{Outil d'Analyse MAQAO}
	\subsection{Definition}
	MAQAO (Modular Assembly Quality Analyzer and Optimizer) est un framework qui permet d’analyser et d’optimiser les performances d’un programme grâce à un ensemble de modules (CQA, LPROF, ONEVIEW). MAQAO effectue une analyse dynamique et statique du code et ainsi détermine les éléments limitant la performance d'une zone de l'application, il travaille au niveau binaire il n’a pas de restriction par rapport au langage utilisé dans le code source. l’objectif principal de MAQAO est de guider les développeurs d'applications tout au long du processus d'optimisation grâce à des rapports synthétiques et des astuces.[w1]

	\subsection{Les modules de base }
    Les pricipaux modules de base qu'utilise MAQAO pour la génération de rapports d'analyse de performance sont :
	\subsubsection{LProf}
	    LPROF (Lightweight Profiler) est un module de MAQAO, c’est un profiler qui vérifie l’état d’un programme en effectuant un échantillonnage (200/s par défaut). Ainsi il permet de détecter le point chaud du programme. Son utilisation est relativement simple elle consiste en une ligne de commande composée de différents champs[w1]:\\ 
	    \begin{itemize}
	        \item Nom de l'application (ou chemin sinon situé dans le répertoire courant). 
	        \item Les arguments de l’application, le cas échéant.
	    \end{itemize}
	    Il existe plusieurs autres options pour LPROF:\\
	    
        \$ maqao lprof – help
	    
            \$ maqao lprof -- $<$application$>$ [arg1 arg2 ...]
	\subsubsection{CQA}
	    CQA (Code Quality Analyzer) est un module de MAQAO, Le module CQA analyse le code généré statiquement et évalue son efficacité sur la base de l’architecture de la machine cible et des registres et instructions utilisées par le programme [w1]. Son utilisation est relativement simple elle consiste en une ligne de commande composée de différents champs:\\
\$ maqao cqa $<$BINARY$\_$FILE$>$ fct-loops=$<$FUNCTION$>$ uarch=$<$MICRO-ARCHITECTURE$>$ \\
\begin{itemize}
    \item Nom du fichier binaire à analyser (ou chemin si non présent dans le répertoire courant).
    \item Nom des fonctions à analyser. Vous pouvez donner une liste d'expressions rationnelles: 	foo,\^bar\$ correspondra à foo29, my$\_$foo et bar mais pas my$\_$bar.
    \item Micro-architectures prises en charge (CORE265 and CORE2$\_$45 (Core 2 65/45 nm), 	NEHALEM (Nehalem and Westmere), SANDY$\_$BRIDGE & IVY$\_$BRIDGE, HASWELL & 	BROADWELL, SKYLAKE (incl. Skylake SP), KABY/COMET$\_$LAKE, ICELAKE \& 	ICELAKE$\_$SP, KNL or KNIGHTS$\_$LANDING, AMD Zen/Zen+/Zen 2/Zen 3).
\end{itemize}
	\subsubsection{ONE View}
	\textbf{Generation de rapport avec Oneview:}\\
	
Oneview est un module délivré par MAQAO, il permet de générer des rapport d’analyse de codes source compilés en se basant sur différent modules. LPROF un profiler basé sur l'échantillonnage,et  CQA un analyseur statique qui évalue la qualité du code généré par le compilateur. Oneview propose deux format de rapport HTML(defaut), XSLX ou text[w1]. Son utilisation est relativement simple elle consiste en une ligne de commande composée de différents champs:\\
 \$ maqao oneview –create-report=$<$report$>$ –c=$<$config$>$ [-xp=$<$dir$>$] [-of=$<$format$>$] [--with-scalability]   
\begin{itemize}
    \item  -create-report=<report> permet de réalisez un rapport et exécutez toutes les étapes nécessaires les valeurs disponibles sont: one.
    \item   –c=$<$config$>$  spécifie le chemin d'un fichier de configuration.
    \item    -xp=$<$dir$>$ spécifie le chemin d'un répertoire d'expérimentation.
    \item -of=$<$format$>$ spécifie le format du rapport généré (HTML pas défaut).
    \item  --with-scalability active l’analyse d’évolutivité.
\end{itemize} 

     

